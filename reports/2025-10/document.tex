\maketitle
%\thispagestyle{fancy}
\section*{Overview}

We continued our research on a planned overhaul to Swarm's staking system, focussing on building a more logical, easy-to-understand structure for the stake registry with a short-term objective of safely enabling stake withdrawals.\footnote{\url{https://github.com/shtukaresearch/swarmonomics/blob/main/2025-research-programme.md}}

Aata returned from his Budapest residency after extensive discussions.

\begin{enumerate}
  \item \emph{Economic roadmap.} Continue work on restructuring the staking system on a more principled, understandable approach, in particular enabling withdrawals.

  We have published a draft SWIP introducing an \emph{stake registry update queue}, which introduces tunable delay parameters for changes to entries in the stake registry.\footnote{\url{https://github.com/ethersphere/SWIPs/pull/79}}
  %
  The core idea is to enable longer delays for changes that \emph{reduce} a node's storage commitment, such as reducing height (controlling the amount of storage committed).
  %
  Combined with a proposal to allow drawdowns of committed stake, it allows us to impose a delay of, say, one month, reducing network churn.

  \item \emph{Budapest residency post-mortem.} Aata completed his residency at the Solar Punk offices.
  %
  We made the following progress on objectives:
  \begin{enumerate}
    \item \emph{Align on goals of BZZ tokenomics.} 
    %
    There were extensive discussions on strategies for activating idle BZZ capital such as liquid staking or other yield-bearing instruments.
    
    Seeking broader alignment, we wrote a memo expressing a new perspective on the value proposition of BZZ and its consequences for BZZ tokenomics and shared it with key foundation personnel.\footnote{\url{https://hackmd.io/@i79XZRmjR86P6AbhL0jwVQ/S19NWmjpge}}
    %
    After responding to initial feedback, we are now sharing this with the wider community for further feedback and engagement.
    \item \emph{Align on protocol improvement roadmap.} 
    %
    We have learned of several contrasting viewpoints on protocol roadmap direction.
    %
    We believe the optimal future direction of Swarm will be based on finding common ground between these viewpoints and using it to build alignment.
    %
    We continue to work on gathering and synthesising viewpoints with a view to aiding the Foundation in aligning its partners and the wider community.

    \item \emph{Discuss selected applications of Swarm storage with a view to market positioning.} 
    %
    We held research meetings on a potential application of Swarm that provides email-like functionality without servers or DNS. A presentation at the 39th Chaos Communication Congress in Hamburg is in preparation.\footnote{\url{https://events.ccc.de/congress/2025/infos/index.html}}
  \end{enumerate}
  
\end{enumerate}

\newpage
\section*{Project management}

\subsection*{Completed Milestones}

\begin{itemize}
  \item Evaluate the benefits and limitations of applying a queue model to migrations or drawdowns.
  \item Write concrete proposal (SWIP) for stake withdrawals with a simple exit queue.
  \item Disseminate thesis document on value proposition for BZZ token as a guiding principle for tokenomics.    
\end{itemize}

\subsection*{Next milestones}

\begin{itemize}
  \item Align Swarm Research and engineering team on 
  \item Triage on migration-free stake registry upgrade approach. 
  \item Write concrete proposal (SWIP) for migration-free stake registry upgrade model.
  \item Look into the benefits and limitations of an inactivity leak penalty.
\end{itemize}

\subsection*{Later milestones}
\begin{itemize}
  \item Statistical and dynamical systems analysis of price oracle.
  \item Empirical study of node strategies.
  \item Study effect of freezing or slashing parameter on payoffs of consensus attacks.
  \item Evaluate threat level of ghettoisation consensus attacks.
  \item List mitigations for ghettoisation consensus attacks.
  \item Mathematical model for address allocation schemes and balancing.
\end{itemize}


\subsection*{Roadmap highlights}

\paragraph{November}
%
Align on stake registry update SWIPs and decide whether we ought to build out the migration-free stake registry upgrade approach first.
%
Publish semiannual report.

\subsection*{Roadmap changes}

We're no longer expecting to finalise any changes to the stake registry in time for Devconnect.


\newpage
\section*{Current work}

We're actively working on multiple research directions as part of a broader effort to revamp Swarm's economic system, now ongoing for the last couple of months.

\paragraph{BZZ tokenomics roadmap}

We sketched out a vision for the investment case for BZZ in a new memo in which \textbf{staking to receive a share of Protocol revenue} is the primary value driver and \textbf{TVL} the primary success metric.
%
Read it at \url{https://hackmd.io/@i79XZRmjR86P6AbhL0jwVQ/S19NWmjpge}.

This investment thesis will help us develop a \textbf{tokenomics roadmap} for BZZ as part of a broader effort to revamp Swarm

\paragraph{Enable withdrawals.}
%
Unlock full stake withdrawal under normal network conditions.
%
Control the speed of withdrawals (and migrations) with a queue.

Queue proposal: \url{https://github.com/ethersphere/SWIPs/pull/79}


\paragraph{Evaluate inactivity leaks}
%
The Swarm Network sometimes suffers from unscheduled node blackouts, threatening network stability. 
%
For example, in late October over 1000 nodes appear to have gone silent for about a week.
%
Clearly, the negative incentive of lost revenue has not been sufficient to incentivise a reliable, higher quality node population.
%
With stake withdrawals enabled, further negative incentives such as bleeding out stake could be applied.

\paragraph{Streamline incentive system upgrade path}
    %
    Separate the upgrade path of stake account management from that of encumbrances (locks) and resource commitment metadata.
    %
    Enable upgrades to participation metadata schema and account locking logic without asset migration.\footnote{For a sketch of an approach, slides from our presentation to Swarm Foundation research can be found at \url{https://assets.super.so/39e7899c-5b59-4143-a573-ea1301d7d10f/files/6403836d-9c7b-46d8-ae76-478b7bf3b8cc/Non-custodial_model_for_Swarm_staking.pdf}.}

    Improve upgrade UX for both protocol engineers and node operators.

    \emph{Dependencies.}
    %
    Careful enumeration of current responsibilities of stake registry.
    %
    Design UX flow for frictionless upgrades.

\paragraph{Simplify node operator strategies}
%
Currently, Swarm node operators face a confusing decision of how much BZZ to stake and into which neighbourhoods.
%
Because neighbourhood populations are small, the mechanism is potentially adversarial — the optimal strategy depends on how the staker believes the other participants in the neighbourhood will respond.

There are several avenues by which we could address this:
%
\begin{enumerate}
  \item Improve tooling.
  %
  Provide node operators with out-of-the-box solutions to run large numbers of nodes and to automate splitting stake up among the wallets of all these nodes.
  %
  If stake can easily be evenly spread amongst hundreds or thousands of neighbourhoods, stake topups have less direct impact on the incomes of competing stakers, reducing the chance of a reaction.

  %
  \item Have the network dictate a fixed amount of stake per node.

  \emph{Advantages.} Clearly simplifies staker decisioning. Optimal strategy is to run as many nodes as you can based on current operating costs (including cost of stake capital) and revenue. Since all nodes are supposed to do the same amount of work and provide the same value, requiring the same amount of collateral to be attached to each node is intuitively reasonable.

  \emph{Disadvantages.} NOs still need to strategise in order to get their nodes allocated into neighbourhoods without overpopulating them. Limits TVL growth per unit demand. Place additional burden on system designers to pick the fixed amount. May create incentive to Sybil via a deduplication attack.

  \item Calculate revenue shares based on global stake share, not per-neighbourhood share.

  \emph{Advantages.} Maximise stake liquidity. For all but the largest stakers, optimal strategy is simply to stake as much BZZ as you want to expose to Swarm Network revenue at the available ROI.

  \emph{Disadvantages.} Auxiliary system needed to compensate node operators for work, link stake collateral to nodes for the sake of applying penalties, and ensure balancing.
\end{enumerate}

\emph{Dependencies.}
%
Research into mechanisms to automate allocation of nodes to neighbourhoods (or allocate overlay addresses directly).
%
Research into balancing properties of node overlay address populations.


\paragraph{Evaluate consensus-level threats}
%
Swarm's method for achieving consensus over the network depth and chunk index depends on a non-standard single-round leader-based algorithm and, because it is weighted on a per-neighbourhood basis, has very low Sybil resistance.
%
We have found potential weaknesses to this consensus system which may threaten desired properties of Swarm Network such as censorship-resistance.
%
Key strategies for improving Sybil resistance are \textbf{aggregating across neighbourhoods} and \textbf{increasing TVL}.
