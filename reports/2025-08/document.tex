\maketitle
%\thispagestyle{fancy}
\section*{Overview}

We continued to build out our stake transparency dashboard, and also began work to support a project to overhaul Swarm's staking system.\footnote{\url{https://github.com/shtukaresearch/swarmonomics/blob/main/2025-research-programme.md}}

\begin{enumerate}
  \item \emph{Alternative stake models.} Begin building a roadmap to a revamped staking system.

  Post a SWIP that provides a simple way to enable stake withdrawals.
  
  PR: \url{https://github.com/ethersphere/SWIPs/pull/77}
  
  \item \emph{Stake transparency.} Visual improvements to our staking dashboard at \url{https://beta.dash.swarm.shtuka.io}.
  %
  API now provides a list of neighbourhoods with their effective stake power, correctly accounting for height (area of responsibility multiplier).

\end{enumerate}


\section*{Project management}

We have revisited our tactical priorities to better align with the current activities of the Swarm core developer and research community.
%
As of last month, the Swarm Foundation research team has begun work on an overhaul of the staking and incentive system.\footnote{\url{https://blog.ethswarm.org/foundation/2025/monthly-development-update-july-2025/}}

\subsection*{Completed Milestones}

\begin{itemize}
  \item 
    Sketch modular framework for stake upgrades, leveraging non-custodial stake model.
  \item
    Improvements and community feedback on Swarm staking dashboard.
  \item
    Author SWIP on enabling withdrawable stake.

\end{itemize}

\subsection*{Next milestones}

\begin{itemize}
  \item Quantify effect of inactivity leak and withdrawal delay parameter on network stability.
  \item Flesh out migration-free stake registry upgrade model.
  \item Analyse economic impact of per node stake limits.
\end{itemize}

\subsection*{Later milestones}
\begin{itemize}
  \item Statistical and dynamical systems analysis of price oracle.
  \item Empirical study of node strategies.
  \item Study effect of freezing or slashing parameter on payoffs of consensus attacks.
  \item Evaluate threat level of ghettoisation consensus attacks.
  \item List mitigations for ghettoisation consensus attacks.
  \item Mathematical model for address allocation schemes and balancing.
\end{itemize}


\subsection*{Roadmap highlights}

\paragraph{September--October}
%
Research support work for staking system revamp.

\paragraph{November}
%
Publication of semiannual report.


\subsection*{Roadmap revisions}

\begin{itemize}
  \item We have revisited our tactical priorities to better align with the current activities of the Swarm core developer and research community.
  %
  As of last month, the Swarm Foundation research team has begun work on an overhaul of the staking and incentive system.\footnote{\url{https://blog.ethswarm.org/foundation/2025/monthly-development-update-july-2025/}}
  %
  We are now prioritising research and SWIP authoring work in support of that goal.
  
  \item Econometric studies and our security work on ghettoisation has been put on hold until more resources become available.
\end{itemize}



\section*{Findings}

\subsection*{Roadmap for improving Swarm's staking mechanism}

The Swarm research community has embarked on a path to revamping Swarm's incentives to better align with participant expectations, streamline onboarding, improve observability, and hand core developers better tools to underwrite network stability.
%
This will likely involve numerous changes over several upgrade deployments.
%
We started mapping out some key research areas where Shtuka Research could provide support in this mission.

\paragraph{Self-custodial stake registry}
    %
    Separate the upgrade path of stake account management from that of encumbrances and participation metadata.
    %
    Enable upgrades to participation metadata schema and account locking logic without asset migration.\footnote{For a sketch of an approach, slides from our presentation to Swarm Foundation research can be found at \url{https://assets.super.so/39e7899c-5b59-4143-a573-ea1301d7d10f/files/6403836d-9c7b-46d8-ae76-478b7bf3b8cc/Non-custodial_model_for_Swarm_staking.pdf}.}

    \emph{Dependencies.}
    %
    Careful enumeration of current responsibilities of stake registry.
    %
    Design UX flow for frictionless upgrades.

\paragraph{Enabling withdrawals.}
    %
    Unlock full stake withdrawal under normal network conditions without enabling exploits in the redistribution game.

    \emph{Dependencies:} 
    %
    Validate tolerance for node churn or stake volatility that may occur as a result of enabling instant withdrawals.
    %
    May require throttling withdrawals.

\paragraph{Throttling withdrawals.}
    %
    Add controls to slow down instant withdrawals and hence node churn, improving network stability.
    
    \emph{Dependencies:} 
    %
    Research to quantify desired throttling effect and how to optimise parameter settings to achieve it.
    %
    Requires negative incentives for unscheduled node inactivity.

\paragraph{System defined stake threshold}
    %
    Introduce system defined stake threshold to reduce decision burden on node operators.

    \emph{Dependencies.}
    %
    Research into economic impact of upper limits on stake deposits; see below.


\subsection*{Stake liquidity}

\emph{Stake liquidity} measures how easy it is to move BZZ tokens in and out of stake positions without substantial impact on per unit returns to stake.

Abstractly, high stake liquidity may be desirable because it provides easy access to a share of network revenue, which in turn provides a fundamental value proposition for holding the token that is directly linked to the success of the network.
%
In the other direction, it allows the market to better absorb shocks such as sharp changes in volume which could affect the desirability of holding liquid BZZ.

For a few reasons, Swarm's current staking system has extremely poor stake liquidity.
%
\begin{itemize}

  \item \emph{Withdrawal restrictions.} Stake can only be withdrawn partially and under special market conditions, so there is essentially no sell-side liquidity (i.e.~no one is selling stake positions for liquid BZZ).
  \item \emph{Operational hurdles.} One must run a node to stake. There is no system in place to delegate stake, or lend BZZ to node operators.
  \item \emph{Adversarial behaviour.}
    %
    Revenue shares are determined in terms of stake per neighbourhood. 
    %
    Since neighbourhoods generally have few residents (the target being 4), adjusting stake has a large impact on the shares of other players and is likely to elicit an adversarial response. 
    %
    In order to maintain good relations with their peers, players may therefore be reluctant to put increased amounts of stake on their node. 
    %
    Consequently, despite being elective by construction, the stake system may end up simulating a fixed stake system with the target being an arbitrary Schelling point.
\end{itemize}
%
These issues could be mitigated by enabling withdrawals more generally, introducing a delegated stake system or BZZ money market, and redesigning the revenue sharing scheme to takes into account global stake share instead of within-neighbourhood share.

The concept of stake liquidity can help us express the economic impact of system-defined stake thresholds.
%
An upper limit on stake per node puts a hard limit on buy-side stake liquidity, limiting access to Swarm cash flows among BZZ holders. 
%
It couples the cost of scaling an investment in Swarm Network cash flows to the cost of adding physical infrastructure, as well as imposing an overall bound in terms of demand.

On the other hand, in a highly liquid elective stake system, the market determines how much stake new operators must put down in order to earn a sufficient share of revenue to cover the network mean operating costs. 
%
This may be restrictive for operators with better access to physical infrastructure than BZZ capital, for example those planning to deploy using spare capacity on hardware they already own.
%
This hurdle could be addressed by introducing a highly accessible BZZ money market for stakers, or a stake delegation system.